\chapter{Увод}

Технолошки развој, а посебно развој интернета, је довео до тога да је интернет постао саставни и готово неизоставни део свакодневног живота, а постојање и широка употреба мобилних уређаја (паметних телефона, нетбук рачунара, таблет рачунара,...) временом је развила потребу за сталним приступом приватним подацима и документима. Самим тим складиштење приватних података и докумената на кућним стоним рачунарима временом је постало превазиђено, а као алтернатива појавило се рачунарство у облаку.

Коришћењем рачунарства у облаку могуће је складиштити личне податке на приватном удаљеном серверу, при том имајући могућност приступа тим подацима са било које локације на интернету, употребом било ког мобилног уређаја, што се у великој мери преклапа са наведеним тенденцијама. Поред великог броја комерцијалних решења, попут \textit{Dropbox-а}, развијена су и многобројна "отворена" решења која корисницима на једноставан и интуитиван начин обезбеђују већу контролу над подацима. Једно од таквих "отворених" решења је и \textit{OwnCloud}. 

Поред основне функционалнсоти, складиштења приватних података, \textit{OwnCloud} нуди и могућност вођења календара догађаја. Основни алат за коришћење свих сервиса које нуди \textit{OwnCloud} је одговарајући веб портал. Поред дате апликације, постоји још доста других апликација (десктоп клијенти, мобилне апликације,...) које \textit{OwnCloud} нуди. Такође, у понуди је и велики број софтвера са "отвореним" лиценцама, иза којих не стоји \textit{OwnCloud} тим, а који нуди услуге које у основи користе \textit{OwnCloud} сервисе. Сва та решења, било да су званичне \textit{OwnCloud} апликације, било да су \textit{3rd party} апликације, суочавају се са одређеним ограничењима. На пример, званични \textit{OwnCloud} десктоп клијент има одвојене верзије за водеће оперативне системе (\textit{Windows}, \textit{Mаc}, \textit{Linux}) i базиран је искључиво на синхронизацију докумената, док, на пример, синхронизација календара догађаја не постоји. 
Тема овог рада, развој мулитиплатформског десктоп клијент који би омогућио синхронизацију календара догађаја, представља један покушај за превазилажење наведених ограничења. Са друге стране, потенцијална употребна вредност таквог решења је још један од мотива за развој датог решења:
\begin{itemize}
	\item{евиденција планираних активности корисника са приказом опдговарајућег обавештења},
	\item{термини полагања испита},
	\item{и било које друге активности које су у основи засноване на дељењу информација о догађајима/активностима измеђе групе корисника система}.
\end{itemize} 
У наставку ће бити укратко описан садржај поглавља овог рада.

Поглавље \textit{Преглед коришћених технологија} представља кратак опис технологија које су коришћене приликом развоја решења које је тема овог рада, док је нешто шири опис дат у поглављу \textit{Радно окружење}.

Поглавље \textit{ownCloud} укратко описује пројекат и апликацију чије сервисе дати десктоп клијент треба да користи. Опис десктоп клијента и приказ кључних делова програмског кода биће представљен у поглављу \textit{ownCloudCalendar}. 