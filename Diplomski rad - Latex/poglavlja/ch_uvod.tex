\chapter{Увод}

Из године у годину стил живота савременог пословног човека се све више усложњава, што доводи до тога да један од основних проблема пословних људи јесте квалитетна организација времена. Како се број обавеза и планираних активности повећава из дана у дан, потреба за неком врстом планера и подсетника се намеће као логична.

Са друге стране, технолошки развој је довео до тога да је интернет постао саставни и готово неизоставни део свакодневног живота, а постојање и широка употреба мобилних уређаја (паметних телефона, нетбук рачунара, таблет рачунара,...) временом је развила потребу за сталним приступом приватним подацима и документима. Самим тим складиштење приватних података и документа на кућним стоним рачунарима полако постаје превазиђено. Као алтернатива намеће се рачунарство у облаку.

Коришћењем рачунарства у облаку могуће је складиштити личне податке на приватном удаљеном серверу, при том имајући могућност приступа тим подацима са било које локације на интернету, употребом било ког мобилног уређаја, што се у великој мери преклапа са наведеним тенденцијама савременог друштва. Поред великог броја комерцијалних решења, попут \textit{Dropbox-а}, развијена су и многобројна "отворена" решења која корисницима на једноставан и интуитиван начин обезбеђују већу контролу над подацима. Једно од таквих "отворених" решења је и \textit{OwnCloud}. 

Поред могућности складиштења приватних података, \textit{OwnCloud} нуди и могућност вођења календара активности, односно неке врсте е-планера. Развој десктоп клијента који би имао функцију подсетника, а који би садржај наведеног календара активности користио као извор података, је тема овог рада. У наставку ће бити укратко описан садржај поглавља овог рада.

Поглавље \textit{Преглед коришћених технологија} представља кратак опис технологија које су коришћене, док је шири опис одговарајућих технологија дат у поглављима \textit{Радно окружење}, \textit{.NET Framework} и \textit{XWT}.

Поглавље \textit{ownCloud} укратко описује пројекат и апликацију чије сервисе дати десктоп клијент треба да користи. Опис десктоп клијента и приказ кључних делова програмског кода биће представљен у поглављу \textit{ownCloudCalendar}. 