\chapter{Преглед коришћених технологија и радно окружење}
\label{chap:Pregled koriscenih tehnologija}

Развој софтверских решења се генерално може поделити на развој комерцијалних софтвера, који се штите власничким лиценцама, и на развој софтвера са нешто "отворенијим" лиценцама, док је програмски код обично отворен и доступан. Када се говори о отвореним лиценцама мора се бити обазрив према отворености и слободи коју лиценца као таква пружа. Примера има пуно, а познате су BSD, GPL и MIT\cite{licence} лиценце, које се користе у свету софтвера отвореног кода.

Постоји неколико различитих алата за контролу изворног кода. До скоро је \textit{Subversion} био најзаступљенији, али у последње време \textit{Git} преузима примат на тржишту, јер је заснован на другачијим принципима, тако да више задовољава потребе корисника. Као такав био је погодан за коришћење и у овом раду заједно са слободним и бесплатним \textit{Git} репозиторијумом \textit{GitHub}, који поред простора који пружа, даје и неопходну статистику везану за број учесника на пројекту, њихову активност итд.

Један од водећих изазова у развоју десктоп апликација је био у томе да се нађе начин за превазилажење ограничења која су изазавана оперативним системима на којима те апликације треба да раде. Разлике у концептима и техничким специфичностима које постоје међу водећим оперативним системима утицале су на то да десктоп апликације развијене за једнм оперативни систем не могу да раде на осталим операривним системима без одговарајућег прилагођавања. Како би се ова ограничења превазишла, јавила се потреба за развојем платформи које ће омогућити да десктоп апликације без проблема раде на свим оперативним системима. Једна од таквих платфотми је и \textit{XWT}\cite{xwt}.

\textit{XWT} је \textit{.NET} мултиплатформски алат за развој корисничког интерфејса. Омогућава развој десктоп апликација које могу да раде на различитим платформама без потребе да се код прилагођава свакој од њих. Разлика у односу на традиционалини приступ у развоју десктоп апликација је у томе што се контроле исцртавају динамички, у самом коду, а сам \textit{XWT API} има способност да у зависности од платформе изабере одговарајуће контроле.

У развоју решења, које је тема датог рада, коришћене су и следеће готове компоненте са отвореним лиценцама:
\begin{itemize}
	\item {\textit{DDay.iCal}\cite{dday} - библиотека класа за рад са календаром за окружење \textit{.NET-a 2.0} и новије верзије},
	\item {\textit{CalDAV}\cite{caldav} - протокол за синхронизацију календара. Представља интернет стандард који омогућава клијенту да приступи информацијама о планираним догађајима на удаљеном серверу. Дозвољава истовремени приступ истим информацијама од стране више клијената, чиме се омогућава кооперативно планирање и дељење информација}.
\end{itemize}