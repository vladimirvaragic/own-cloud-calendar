\chapter{Преглед коришћених технологија}
\label{chap:Pregled koriscenih tehnologija}

У глобалу, сва софтверска решења можемо поделити у две категорије. Једну групу чине комерцијална решења, која су заштићена власничким лиценцама, док другу групу чине софтверска решења са, у већој или мањој мери, "отвореним" лиценцама, при чему је програмски код обично отворен и доступан. У свету софтвера "отвореног" кода постоји више различитих типова лиценци, а неке од познатијих су BSD, GPL и MIT\cite{licence} лиценце. Када се говори о "отвореним" лиценцама мора се бити веома обазрив у смислу отворености и слободе коју лиценца као таква пружа. У складу са тим, потребно је напоменути да је решење које је тема овог рада у потпуности отворено и да је код доступан у целости. 

Софтверски производи су временом постали све сложенији и све компликованији, што је довело до тога да један човек углавном не може сам да се бави имплементацијом неког софтверског решења у прихватљивом временском периоду, већ су на развоју софтвера најчешће ангажовани тимови људи. Из тог разлога, као нов изазов појавила се потреба за решењем које би омогућило да сви чланови тима могу паралелно да раде на развоју софтвера, не угрожавајући активности осталих чланова тима. Као одговор на наведени проблем, појавили су се различити алати за контролу изворног кода. Дужи временски период \textit{Subversion} је био најзаступљенији алат за контролу изворног кода, али у последње време \textit{Git}\cite{git} преузима примат, јер је заснован на другачијим принципима, тако да више задовољава потребе корисника. Као такав \textit{Git} је био погодан за коришћење и у овом раду заједно са слободним и бесплатним \textit{Git} репозиторијумом \textit{GitHub}\cite{github}, који поред простора који пружа, даје и неопходну статистику везану за број учесника на пројекту, њихову активност итд.

У суштини, подела софтверских решења се може  вршити по различитим карактеристикама: оперативним системима за које су развијани, пословним процесима које покривају, технологијама које су коришћене у току развоја итд. Једна од битних подела, која се огледа у потпуно различитим концептима на којима су апликације засноване, јесте подела на десктоп и веб апликације. У смислу наведене поделе, решење које је тема датог рада предстаља десктоп апликацију. Постојање више различитих корисничких интерфејса или више верзија апликације за различите оперативне системе може се сматрати стандардом. С тим у вези, у процесу развоја софтвера, појавила се и потреба за поделом логике и архитектуре уређења самог софтверског решења. На овај начин обезбеђена је оптимизација програмског кода, као и могућност поновне употребе постојећег кода. У складу са наведеном поделом дато решење можемо посматрати као двослојну апликацију, где један слој представља сам кориснички интерфејс, а други слој је задужен за "комуникацију" са \textit{OwnCloud} платформом. 

Један од водећих изазова у развоју десктоп апликација је да се нађе начин за превазилажење ограничења која су изазавана оперативним системима на којима те апликације треба да раде. Разлике у концептима и техничким специфичностима које постоје међу водећим оперативним системима утицале су на то да десктоп апликације развијене за један оперативни систем не могу да раде на другим операривним системима без одговарајућег прилагођавања. Како би се ова ограничења превазишла, јавила се потреба за развојем платформи које би омогућиле да десктоп апликације без проблема раде на свим оперативним системима. Једна од таквих платфотми је и \textit{XWT}\cite{xwt}, о којој ће бити више речено у поглављу \textit{3. Радно окружење}.

У развоју решења, које је тема датог рада, коришћене су и следеће готове компоненте са "отвореним" лиценцама:
\begin{itemize}
	\item {\textit{Log4Net} - библиотека класа за логовање грешака},
	\item {\textit{DDay.iCal}\cite{dday} - библиотека класа за рад са календаром за окружење \textit{.NET-a 2.0} и новије верзије},
	\item {\textit{CalDAV}\cite{caldav} - протокол за синхронизацију календара, који је такође детаљније описан у поглављу \textit{3. Радно окружење}}.
\end{itemize}