\chapter{Закључак}

Иако постоји и примењује се више од 10 година, концепт рачунарства у облаку још увек није на свом врхунцу и сигурно је да представља будућност ИТ услуга. Проблеми и ограничења са којима се овај концепт суочава, као што су сигурност и  доступност, утичу на то да заступљеност на тржишту и даље није на очекиваном нивоу. Рачунарство у облаку има перспективу зато што принципи на којима се заснива, као што је повећавање капацитета и могућности без додатних инвестирања у инфраструктуру, обуке новог особља и куповања додатних лиценци, има пуно предности у односу на досадашње концепте. Да би достигао свој пун потенцијал, мора се још радити на његовом развоју и стварању нових иновација.

Рад на овој теми помогао ми је да детаљније сагледам концепте рачунарства у облаку. Такође, коришћењем \textit{XWT} платформе, сусрео сам се и са другачијим приступом у развоју десктоп апликација, који је у основи општији од стандардног начина за развој \textit{Windows} апликација.

\section{Идеје за даљи развој}

Иако је функционално исправна, постојећу верзију апликације треба посматрати само као полазни корак у развоју коначног производа. Актуелна верзија апликације има својеврсна ограничења условљена коришћеним API-има (нпр. немогућност синронизације више календара истовремено). Унапређења датих API-a или појава нових утицали би на то да се појави потреба за имплементацијом додатних или изменом постојећих функционалности. Са друге стране, постојећа верзија се такође може унапредити на више начина:
\begin{itemize}
	\item{побољшање корисничког интерфејса},
	\item{предефинисани прикази догађаја (за разлику од актуелног приказа свих догађаја)},
	\item{...}
\end{itemize}